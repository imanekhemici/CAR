				
								TP1: implémenter un serveur FTP

	1)Auteurs:
		-KHEMICI Imane
		-TAIFOUR Mina
		-Mbabazi Simbi Raissa

	2)Compilation:
               
		- L'éxécution du projet se fait dans un terminal par la commande: java -jar dossier/ftpServer.jar
		- On lance un autre terminal et on execute la commande: ftp localhost 1024
		- Afin de s'authentifier : 

                login: utilisateur
                mot de passe : mdp

                On peut se connecter sans mot de passe en utilisant :
                login : anonymous
		
	3)Introduction: 
			
		Notre Tp a pour objectif de créer un serveur FTP qui permet de transferrer des fichiers d'une machine à une autre; en utilisant le port 1024, afin de communiquer avec des clients,
		le serveur gère les requêtes USER, PASS, QUIT, LIST , RETR, PORT, PWD, CWD directory.
	
	4) Architecture du code: 

       *Package ftp: 

		- La class "ftpServer": qui contient la fonction principale main(), qui fait l'initialisation du serveur ftp avec le port, puis elle execute le process
		ftpRequest créé.
		
		- La classe "FTPRequest": contient les méthodes qui traitent les commandes suivantes:
			-- USER , processUSER()
			-- PASS, processPASS()
			-- RETR, processRETR()
			-- STOR, processSTOR()
			-- LIST, processLIST()
			-- QUIT, processQUIT()
			-- PWD,  processPWD()
                        -- CWD,  processCWD() 
                        Elle contient aussi: 
			-- un constructeur FTPRequest,  
			-- la méthode run() : permet d'interagir client - serveur
			-- la méthode processRequest() : permet de lancer une commande
			-- la méthode processPORT() : permet de specifier le port 
			-- la méthode processSYST() : permet d'afficher le systeme d'exploitation utilisé
                        -- la méthode envoyerFichier(): permet d'envoyer un fichier du serveur vers le dossier local
                        -- la méthode recoisFichier(): permet de recevoir un fichier sur notre serveur a partir d'un dossier local
                        -- la méthode sendMessage(): permet d'envoyer un message en octet
                        -- la méthode sendListe(): permet de recuperer la liste des fichiers dans un dossier 


		- la classe FTPRequestTest qui contient les tests concernant la connection et la déconnextion de l'utilisateur, ainsi que le test sur le  mot de passe.
        *Package ftpTest:
                - la classe FTPRequestTest contient les tests de nos methodes:

                        cette classe permet de tester la connexion à un serveur, la deconnexion et l'authentification.
                
			
	4)Compatibilité

		Nous avons tester notre serveur seulement avec la commande ftp. Il n'est pas compatible (à priori ) avec filezilla.	

	5)Remarques
	       La commande CDUP n'a pas été implementé.
               Les tests des autres methodes n'ont pas été implementé.
               Notre code a été insperé de d'autres codes trouvés sur interne (Site comme le site de zero, developper et etc)
               Afin de visioner le code sous eclipse il faut importe le dossier tp1.
               L'utilisateur peut acceder à tout les dossier mais l'anonymous.
